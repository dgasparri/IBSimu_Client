\documentclass[12pt,a4paper]{article}
\usepackage{color}
\usepackage{parskip}% http://ctan.org/pkg/parskip

% \setlength{\parindent}{0pt}

\title{IBSimu Particle Diagnostic \\
	\large Working Draft}
\date{2020-11-30}
\author{Duccio Marco Gasparri}

\begin{document}
  \maketitle
  
\section{variables}
\begin{itemize}
	\item m [kg] particle mass (provided in u)
	\item q [J] charge of beam particle (provided in multiples of e)
	\item J [A/m\^2] beam current density
	\item E [J]  mean energy (provided in eV)
	\item Tp [J] parallel temperature (provided in eV)
	\item Tt [J] transverse temperature (provided in eV)
	\item ($x_{1}, r_{1}), (x_{2}, r_{2}$) [m] beam emission line vectors
	\item N number of particles
	\item IQ [A] (A/m?) particle current
	\item v [m/s] from E and m
\end{itemize}

\section{Setup}
\subsection{Geometry}

Relevant IBSimu files:
\begin{itemize}
	\item geometry.hpp 
	\item geometry.cpp
	\item mesh.hpp
	\item mesh.cpp
\end{itemize}



Relevant client files:
\begin{itemize}
	\item configuration-setup.hpp
	\item configuration-setup.cpp
\end{itemize}


Parameters:
\begin{itemize}
	\item mesh-cell-size-h [eg. 0.5e-4]
	\item origin-x 
	\item origin-y
	\item origin-z
	\item geometry-start-x [UNUSED]
	\item geometry-start-y [UNUSED]
	\item geometry-start-z [UNUSED]
	\item geometry-size-x
	\item geometry-size-y
	\item geometry-size-z
\end{itemize}

The class Geometry creates the mesh of the simulation starting from the CAD files and/or function definitions.

The parameters origin-x, origin-y and origin-z are used to move the reference frame relative to the bbox in the DXF file [see mesh.hpp line 70-73, mesh.cpp line 60].

The physical length/height/depth of the simulation is size-x/y/z * mesh-cell-size. Specifically:

$$size-x = (int)floor(geometry-size-x / mesh-cell-size-h) + 1 $$
$$x_{max} = origin-x + mesh-cell-size-h * size-x$$







\section{ParticleDataBaseCylImp::add\_2d\_beam\_with\_energy}

Function ParticleDataBaseCylImp::add\_2d\_beam\_with\_energy (file: \textit{particledatabaseimp.cpp}, line: 968) is used to add a beam of N particles with average energy E to a cylindrical geometry.

The charge q is provided by the user and is set constant for all the particles.

The beam emission line norm s [m] is defined:

\begin{equation}
	s = \sqrt{(x_{2}-x_{1})^2+(r_{2}-r_{1})^2}
\end{equation}

The current IQ [A] is set for each particle as follows:

\begin{equation}
	IQ =\frac{2\pi sJ}{N}(r_{1} + \frac{(r_{2}-r_{1})}{N}(n+0.5))
\end{equation}

where $n\in[0,1,...,N-1]$. 

The particles are distributed evenly spaced along the emission line defined by the vectors $(x_{1},r_{1}), (x_{2},r_{2})$. The particle velocities $v_{x},v_{r}$ [m/s] are:

\begin{equation}
	v_{x}=\frac{(x_{2}-x_{1})}{s}\sqrt{\frac{Tt}{m}}rnd_{0}+\frac{(r_{2}-r_{1})}{s}\sqrt{\frac{2E}{m}+(\sqrt{\frac{Tp}{m}}rnd_{1})^2}
\end{equation}

\begin{equation}
	v_{r}=\frac{(r_{2}-r_{1})}{s}\sqrt{\frac{Tt}{m}}rnd_{0}+\frac{-(x_{2}-x_{1})}{s}\sqrt{\frac{2E}{m}+(\sqrt{\frac{Tp}{m}}rnd_{1})^2} 
\end{equation}

and

\begin{equation}
	w=\frac{d\theta}{dt}=\frac{\sqrt{\frac{Tt}{m}}rnd_{2}}{r_{1} + \frac{(r_{2}-r_{1})}{N}(n+0.5)}
\end{equation}

with $rnd_{0}$, $rnd_{1}$ and $rnd_{2}$ normally distributed random variables.


\section{Iteration}

The method ParticleDataBasePPImp::iterate\_trajectories (inherits ParticleDataBaseImp, file: particledatabaseimp.hpp::641) outputs the message "Using non-relativistic iterator", clears the scharge, creates a iterators vector of size equal to the set thread, populates it with new instances of ParticleIterator<PP> and waits for the iteration to be finished. Then calls scharge\_finalize\_linear (file:scharge.cpp:149) Finally it publishes the particle hisories ("Particle histories").

scharge\_finalize\_liear prints "Finalizing space charge density map (LINEAR method)"

The core of the iteration isinside the ParticleIterator<PP>


\section{Particle Diagnostic}

The relevant functions are in files gtkparticlediagdialog.cpp (the GTK dialog file) and particlediagplot.cpp (does the actual plotting). It has the following methods:

\begin{itemize}
	\item ParticleDiagPlot::build\_data(): the function extracts the data from the ParticleDatabase
	\item ParticleDiagPlot::build\_plot(): calls build\_data(). the function extracts the data from the ParticleDatabase, set the decorations and add the graph to the \_frame
\end{itemize}


Trajectories are obtained from ParticleDataBase::trajectories\_at\_plane() that calls ParticleDataBaseImp::trajectories\_at\_plane(). It requires the axis (X, Y, Z, R) and the distance from origin. It returns a vector of particle point objects ParticleP<PP> (either ParticleP2D, ParticlePCyl, or ParticleP3D), each particle carrying its information about position and momentum

q is the direction normal to the diagnostic plane

DIAG\_X: x position [m]
DIAG\_R: r position [m]



DIAG\_RP: $\frac{v_r}{v_q}$

DIAG\_AP: $\frac{v_\theta}{v_q}$

$$v_\theta=r\frac{d\theta}{dt}$$

Emittance:
\begin{itemize}
	\item X axis, r-r': DIAG\_R (r position), DIAG\_RP ($\frac{v_r}{v_q}$)
	\item X axis, r-a': DIAG\_R (r position), DIAG\_AP ($\frac{v_\theta}{v_q}$)
	\item X axis, z-z': using DIAG\_R, DIAG\_RP, DIAG\_AP and EmittanceConv class instead
\end{itemize}


The diagnostic data are obtained from the Emittance class. For Cylindrical coordinates, the class EmittanceConv converts to (x,x') equivalent emittance from the following data:

\begin{itemize}
	\item r (radius)
	\item rp (radial angle)
	\item ap (skew angle)
	\item I (current)
\end{itemize}

It builds (x,x') data in a grid array of size n by m. Here the skew angle is $ r\omega/v_z $, where $ v_z $ is the velocity to the direction of beam propagation. 



This description from class reference list is not clear: 

The conversion is done by rotating each trajectory diagnostic point around the axis in rotn steps (defaults to 100). The output grid size can be forced by setting (xmin,xpmin,xmax,xpmax) variables, otherwise the grid is autotomatically sized to fit all data.

The emittance statistics is built using original data and not the gridded data for maximized precision.


\section{ToDo}

Particle Types

template<class PP> class Particle \{
	std::vector<PP> 	\_trajectory; <- trajectories
       PP				\_x; <- current position
\}
Typedef Particle<ParticleP2D> Particle2D;
Typedef Particle<ParticlePCyl> ParticleCyl;
Typedef Particle<ParticleP3D> Particle3D;

Particle Types in Beam

ParticleDataBaseCylImp::add\_*\_beam -> ParticlePCyl->ParticleCyl
ParticleDataBase2DImp::add\_*\_beam->ParticleP2D->Particle2D
ParticleDataBase3DImp::add\_*\_beam->ParticleP3D->Particle3D



ParticleDataBasePPImp< PP >::trajectories\_at\_plane 


Int ParticleP2D::trajectory\_intersections\_at\_plane
( NON CONST	std::vector< ParticleP2D > \& 	intsc)
Return the number of trajectory intersections with plane
Intersection points are appended to vector intsc.
Int TrajectoryRep1D::solve()
Returns solutions found [ Linear [0,1], Quadratic[0,1,2], Cubic [0,1,2,3]]

IBSimu DXF bug in mydxffile.cpp \#define CODE\_STRING(x) code (x) == 101 is not included as it is a later standard in DXF file.



\end{document}