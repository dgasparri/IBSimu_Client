\documentclass[12pt,a4paper]{article}
\usepackage{color}
\usepackage{parskip}% http://ctan.org/pkg/parskip

% \setlength{\parindent}{0pt}

\title{IBSimu Add 2d beam}
\date{2020-10-06}

\begin{document}
  \maketitle
  
\section{variables}
\begin{itemize}
	\item m [kg] particle mass (provided in u)
	\item q [J] charge of beam particle (provided in multiples of e)
	\item J [A/m\^2] beam current density
	\item E [J]  mean energy (provided in eV)
	\item Tp [J] parallel temperature (provided in eV)
	\item Tt [J] transverse temperature (provided in eV)
	\item ($x_{1}, r_{1}), (x_{2}, r_{2}$) [m] beam emission line vectors
	\item N number of particles
	\item IQ [A] (A/m?) particle current
	\item v [m/s] from E and m
\end{itemize}

\section{ParticleDataBaseCylImp::add\_2d\_beam\_with\_energy}

Function ParticleDataBaseCylImp::add\_2d\_beam\_with\_energy (file: \textit{particledatabaseimp.cpp}, line: 968) is used to add a beam of N particles with average energy E to a cylindrical geometry.

The charge q is provided by the user and is set constant for all the particles.

The beam emission line norm s [m] is defined:

\begin{equation}
	s = \sqrt{(x_{2}-x_{1})^2+(r_{2}-r_{1})^2}
\end{equation}

The current IQ [A] is set for each particle as follows:

\begin{equation}
	IQ =\frac{2\pi sJ}{N}(r_{1} + \frac{(r_{2}-r_{1})}{N}(n+0.5))
\end{equation}

where $n\in[0,1,...,N-1]$. 

The particles are distributed evenly spaced along the emission line defined by the vectors $(x_{1},r_{1}), (x_{2},r_{2})$. The particle velocities $v_{x},v_{r}$ [m/s] are:

\begin{equation}
	v_{x}=\frac{(x_{2}-x_{1})}{s}\sqrt{\frac{Tt}{m}}rnd_{0}+\frac{(r_{2}-r_{1})}{s}\sqrt{\frac{2E}{m}+(\sqrt{\frac{Tp}{m}}rnd_{1})^2}
\end{equation}

\begin{equation}
	v_{r}=\frac{(r_{2}-r_{1})}{s}\sqrt{\frac{Tt}{m}}rnd_{0}+\frac{-(x_{2}-x_{1})}{s}\sqrt{\frac{2E}{m}+(\sqrt{\frac{Tp}{m}}rnd_{1})^2} 
\end{equation}

and

\begin{equation}
	w=\frac{d\theta}{dt}=\frac{\sqrt{\frac{Tt}{m}}rnd_{2}}{r_{1} + \frac{(r_{2}-r_{1})}{N}(n+0.5)}
\end{equation}

with $rnd_{0}$, $rnd_{1}$ and $rnd_{2}$ normally distributed random variables.

\section{Particle Diagnostic}

The relevant functions are in files gtkparticlediagdialog.cpp (the GTK dialog file) and particlediagplot.cpp (does the actual plotting). It has the following methods:

\begin{itemize}
	\item ParticleDiagPlot::build_data(): the function extracts the data from the ParticleDatabase
	\item ParticleDiagPlot::build_plot(): calls build_data(). the function extracts the data from the ParticleDatabase, set the decorations and add the graph to the _frame
\end{itemize}


\section{ToDo}

Particle Types

template<class PP> class Particle {
	std::vector<PP> 	_trajectory; <- trajectories
       PP				_x; <- current position
}
Typedef Particle<ParticleP2D> Particle2D;
Typedef Particle<ParticlePCyl> ParticleCyl;
Typedef Particle<ParticleP3D> Particle3D;

Particle Types in Beam

ParticleDataBaseCylImp::add_*_beam -> ParticlePCyl->ParticleCyl
ParticleDataBase2DImp::add_*_beam->ParticleP2D->Particle2D
ParticleDataBase3DImp::add_*_beam->ParticleP3D->Particle3D



ParticleDataBasePPImp< PP >::trajectories_at_plane 


Int ParticleP2D::trajectory_intersections_at_plane
( NON CONST	std::vector< ParticleP2D > & 	intsc)
Return the number of trajectory intersections with plane
Intersection points are appended to vector intsc.
Int TrajectoryRep1D::solve()
Returns solutions found [ Linear [0,1], Quadratic[0,1,2], Cubic [0,1,2,3]]





\end{document}