% IBSimu Class diagram
% Author: Duccio Marco Gasparri
% From a sample of Remus Mihail Prunescu https://texample.net/tikz/examples/class-diagram/
\documentclass{minimal}
\usepackage[a4paper,margin=1cm,landscape]{geometry}
\usepackage{tikz}

%%%<
\usepackage{verbatim}
\usepackage[active,tightpage]{preview}
\PreviewEnvironment{tikzpicture}
\setlength\PreviewBorder{5pt}%
%%%>

\begin{comment}
:Title:  Class diagram

\end{comment}
\usetikzlibrary{positioning,shapes,shadows,arrows}

\begin{document}
\tikzstyle{abstract}=[rectangle, draw=black, rounded corners, fill=blue!40, drop shadow,
        text centered, anchor=north, text=white, text width=3cm]
\tikzstyle{comment}=[rectangle, draw=black, rounded corners, fill=green, drop shadow,
        text centered, anchor=north, text=white, text width=3cm]
\tikzstyle{myarrow}=[->, >=open triangle 90, thick]
\tikzstyle{line}=[-, thick]
        
\begin{center}
\begin{tikzpicture}[node distance=2cm]
    \node (AuxNode01) [text width=4cm] {};
    \node (Field) [abstract, rectangle split, rectangle split parts=2, left=of AuxNode01]
        {
            \textbf{Field}
            \nodepart{second}nil
        };
    \node (Mesh) [abstract, rectangle split, rectangle split parts=2, right=of AuxNode01]
        {
            \textbf{Mesh}
            \nodepart{second}nil
        };
    \node (AuxNode02) [text width=0.5cm, below=of Field] {};
    \node (VectorField) [abstract, rectangle split, rectangle split parts=2, left=of AuxNode02]
        {
            \textbf{VectorField}
            \nodepart{second}nil
        };
    \node (ScalarField) [abstract, rectangle split, rectangle split parts=2, right=of AuxNode02]
        {
            \textbf{ScalarField}
            \nodepart{second}nil
        };
        
    %\node (AuxNode03) [text width=0.5cm, below=of VectorField] {};
    \node (EpotEfield) [abstract, rectangle split, rectangle split parts=2, below=of VectorField] %AuxNode03, xshift=2cm
        {
            \textbf{EpotEfield}
            \nodepart{second}nil
        };
        
    \node (AuxNode04) [below=of Mesh] {};
    \node (Geometry) [abstract, rectangle split, rectangle split parts=2, right=of AuxNode04] %, xshift=-2cm
        {
            \textbf{Geometry}
            \nodepart{second}nil
        };
%    \node (AuxNode05) [left=of AuxNode04] {};
%    \node (EpotField) [abstract, rectangle split, rectangle split parts=2, below=of AuxNode05] %, xshift=2cm
    \node (AuxNode05) [below=of ScalarField] {};
    \node (EpotField) [abstract, rectangle split, rectangle split parts=2, right=of AuxNode05] %, xshift=2cm
        {
            \textbf{EpotField}
            \nodepart{second}nil
        };
     
    \draw[myarrow] (VectorField.north) -- ++(0,0.8) -| (Field.south);
    \draw[line] (VectorField.north) -- ++(0,0.8) -| (ScalarField.north);
    
    \draw[myarrow] (EpotEfield.north) -- ++(0,0.8) -| (VectorField.south);
    \draw[myarrow] (Geometry.north) -- ++(0,0.8) -| (Mesh.south);
    \draw[myarrow] (EpotField.north) -- ++(0,0.8) -| (ScalarField.south);
    \draw[line] (EpotField.north) -- ++(0,0.8) -| (Mesh.south); %Added
    %\draw[line] (EpotField.north) -- ++(0,0.8) -| (Geometry.north);
    %\draw[myarrow] (Coolant.east) -- ++(0.2,0) -- ([yshift=0.5cm, xshift=0.2cm] Valve.north east) -|
        
        
\end{tikzpicture}
\end{center}
\end{document}